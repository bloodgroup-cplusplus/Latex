\documentclass{article}
\usepackage[utf8]{inputenc}
\usepackage{amsmath}
\usepackage{amssymb}
\usepackage{graphicx}
\usepackage[pdftex]{pict2e}
\usepackage[dvipsnames]{xcolor}
\setlength{\unitlength}{1cm}
\setlength{\fboxsep}{4pt}
\usepackage{hyperref}
\usepackage{subfig}
\usepackage{tikz} \graphicspath{ {./images/} }


\begin{document}

\section*{ MATHEMATICS OF BITCOIN BLOCKCHAIN }
\hrule
\bigskip

\textbf{Learning the core maths behind anything has always been a daunting task. The number of volunteers in any curriculum be it academics or any informal one undergo a slight hesitation  when it comes to learning the "math behind".}
\\
\\
\textbf{This booklet however aims to mitigate the usual panic  by trying to make the underlying  maths of the Bitcoin Blockchain  as boiled down and simple  as possible.The idea here is to simplify the concepts to a level which can be understood even by a middle school pupil}
\\
\\
\textbf{Before jumping right into the math,it would be unfair to not  mention  the giants who let us stand on their shoulders to see further on the subject matter}
\\
\textbf{ The personalities mentioned below are not in any specific order. The entire booklet is a very simple boiled down version of various works of the giants whose names are cited below. They've always been kind enough to share their immense understanding about the space   in the form of books,blogs,interviews, videos, articles, podcasts etc. Please visit the links that have been mentioned here for further enlightenment. }
\\
\\
\textbf{[1] \url{https://bitcoin.org/en/bitcoin-paper} Satoshi Nakamoto }
\\
\textbf{[2] \url{https://programmingbitcoin.com/} Jimmy Song \quad \url{https://twitter.com/jimmysong}}
\\
\textbf{[3] \url{https://www.3blue1brown.com/} Grant Sanderson \quad \url{https://twitter.com/3blue1brown}}
\\
\textbf{[4] \url{https://vitalik.ca/} Vitalik Buterin \quad 
 \url{https://twitter.com/VitalikButerin}}
\\
\textbf{[5] \url {https://balajis.com/} Balaji Srinivasan \quad  \url{https://twitter.com/balajis}}
\\
\textbf{[6] \url{https://cdixon.org/} Chris Dixon \quad
    \url{https://twitter.com/cdixon}}
\\
\textbf{[7] \url{https://t.co/WwpNlega0K}  Dan Boneh \quad 
\url{https://twitter.com/danboneh} }
\\
\textbf{[8] \url {http://gavwood.com/} Gavin Wood \quad
\url {https://twitter.com/gavofyork} }
\\
\textbf{[9] \url{https://research.cloudflare.com/people/nick-sullivan}
Nick Sullivan \quad \url{https://twitter.com/grittygrease}}
\\
\textbf{[10]\url {https://aantonop.com/} Andreas M. Antonopoulos \quad 
\url{https://twitter.com/aantonop}}
\\
\textbf{[11] \url{https://https://gavinthink.blogspot.com/} Gavin Anderson \quad \url{https://twitter.com/gavinandresen}}
\\
\textbf{[12] \url{https://haseebq.com/} Haseeb Qureshi \quad \url{https://twitter.com/hosseeb}}
\\
\textbf{[13] \url{https://socratica.com/} Socratica \quad \url{https://twitter.com/socratica}}

\pagebreak

\pagebreak
\section * {INTRODUCTION}
\hrule 
\bigskip
\textbf{Let's rewind our journey of reading and writing to a decade or two ( depending on the readers age).We might recall our days in kindergarten where we all were introduced to basic latin alphabets A,B,C or taught counting 
    starting from 0 and gradually to bigger numbers.A few years later we were taught about how words were formed via these letters and how basic arithemetic could be carried out on these digits. Fast forward to now we have become lighting fast readers and writer of  text, documents, messages and evolved as a pretty decent adders, multipliers and subtractors 
}
\\
\textbf{This multi-part ebook tries to establish the same learning pattern for understanding the core math behind blockchain ( bitcoin blockchain). Firstly we learn the A B C's which include understanding what sets are how arithemetics are carried out on a set followed by algebric structures(groups and rings)  and their properties 
Geometry is used when we prove the properties of Groups whereas a slight mathematical rigour is added for proving the same for Fields. This is make sure that readers are introducted to different prespectives of understanding a single concept.Any  mathematical rigor which is used in proofs  is suprisingly  build up upon arithmetics that we've learnt in our middle school days. The book series is divided into 20 pages each. The reason for doing this is to ensure that you read it to completion and give yourself a pat on your back once you complete it rather than getting agitated by the size of it.
In the next section we shall start with our first concept Sets which could be thought as the protagonist of this edition of the ebook. Everything revolves around sets ,operations and properties.
}
\\
\textbf{ Outmost care has been given to make sure that the explanations have mitigated  "mathematical abstractions" making it an enjoyable read to people from all fields. Any mathematical proofs are proven in a way which builds up from basic arithmetic so that the reader can fully understand the proof on a few circle-backs and iterations if not at a single glance. }
\\
\textbf {The ebook is curated in a way that the readers read this ebook followed by upcoming with ease  so that their idea behind the blockchain (bitcoin blockchain) and the mechanics of what is going on gets very clear.}
\\
\pagebreak
\section*{SETS}
\hrule 
\bigskip
\textbf{Imagine we the readers are travelling in a spaceship and  have been assigned a huge container  to fit all the planets.Initially our container looks something like this \{\} as empty as it can get . 
\\
Now we start collecting planets in our container one after another. First we grab Mercury inside our container(since it's small and easy to capture).The container which was empty before now looks like this \{"Mercury"\}. As we move further we add Venus into it. Our container now looks like \{"Mercury, "Venus"\} now we add Earth. 
As we were adding Earth into the container we realised that the planets inside our container aren't in the same order that we added it them. On adding earth we see our container looks something like this \{"Earth","Mercury", "Venus"\}. As we were roaming in a spaceship we found a twin brother of Mercury and tried to add it but as we were trying to add it we realised that only one of the Mercury either the Mercury we had earlier or the twin brother of Mercury could be added as our container didn't allow us to add both of the same things.  Finally after  adding all eight planets into our container we finally saw that our container looked something like this 
\{"Mercury,"Jupiter","Earth","Mars","Venus","Saturn","Uranus","Neptune"\}
We studied few of the characteristics of our container and wrote reported it as:
\\ 1) Our container didn't allow duplicate elements..
\\ 2) Our container didn't bother the order which objects were placed.
\\ 3) Our container could be empty or it could fit as many unique objects (here we have 8 ) as it could.\\Mathematically,a "set" is something which contains or members, which can be  objects of any kind: numbers, symbols, points in space, lines, other geometrical shapes, variable,planets, or it could just be empty. In our case we we have a finite set containing eight planets, The set could have been infinite had our set had all counting numbers starting from now but for now we keep planets in it.}

\pagebreak 
\section * {ALGEBRIC STRUCTURES}
\hrule
\bigskip
\textbf{On the previous page we looked at sets which was collection of any unordered objects.Our set contained the eight planets of the solar system. In reality these sets could contain not only planets from the solar system but also  numbers,names,points,objects,or any other  collections.One important characteristics of such set is that if we take any set that is not empty  and perform certain operations such as addition, multiplication,subtraction (binary operations) on the objects or elements present in that set we see that  they inherently satisfy or preserve certain set of fundamental statements or propositions that are self-evidently true.In mathematical language these propositions are called as axioms.We might be very aware of general propositions
like   "The sun always rises in the east and sets in the west".Similarly there are propositions in mathematics which are always preserved. (mathematically such propositions which are by default true or established or accepted are known as  "axioms" ) 
\\
This is the right time we define what algebraic structures actually are : Algebraic structures are categorization of sets based on operations such as addition,multiplication that is performed on those set elements and the mathematical propositions they preserve upon carrying out those operations on elements. The field of math which studies this is known as Abstract Algebra.  One might ask why do we even care about propositions out of these sets based on operations.Well there could be a lot of reasons but a few of the most important reason for classification of sets based on operations  is to draw generalizations among systems and sets and to check how which sets satisfy and preserve which proposition while any operation like addition,subtraction,multiplication is applied on its elements,the second reason could be discover new systems with similar properties and decide their categories and prove theorems about all systems with the same basic properties.}
\\
\textbf{ Here we shall be discussing about two common algebraic  structures which lay an important foundation for understanding the mathematics behind the bitcoin blockchain the two of them include Groups and Fields. Understanding Groups and Fields (finite fields) and the propositions their objects or elements preserve and  establish  under binary operation like addition and multiplication helps us understanding elliptic curves which are the heart of how transactions are signed using keys in a bitcoin blockchain.We shall now discuss the first algebraic structure i,e., Groups}
\pagebreak
\section * {GROUPS ( G, . )}
\hrule 
\bigskip
\textbf{ The first algebraic structure we study here are Groups. Groups (mathematically denoted by G) are finite or infinite sets which on performing some binary operation ( binary operation includes '+','-','*','/' ) on its elements preserve four fundamental propositions of closure,associativity,identity and inverse.The mathematical notation of groups i.e., ( G, . ) can be understood by replacing the '.' in ( G, . ) by any arithmetic operator.Let's add '+' operator in it . (G,'+') could be understood as a set which satisfies and preserves the four proposition of groups(identity,associativity,inverse,closure) when any two elements from the set is taken and addition is performed on its elements or objects .Similarly ( G, * ) satisfies and preserves the four group propositions under multiplication of  its elements 
Later we shall look at Fields which are second algebraic structures which preserve these propositions  for both addition and multiplication(G,'*','+') when opertaions are applied on it's elements.}
\\
\textbf{ We shall now understand what actually are these four propositions (closure,associativity,identity and inverse ) and how do these make sense geometrically.
For understanding this clearly let's take a square. Since middle we were taught to find out the area of square or perimeter of the square but here we won't bother finding any of it rather we take the square make it dance a bit . Each of the square's dance move shall be denoted by a symbol.}
\pagebreak
\textbf{Our square here is quiet flexible with a total of eight dance moves.Mathematically these dance move of any objects are referred to as "symmetries"  of it but we shall stick to "dance moves" \\ .We now introduce the moves of the square one by one.The move where our square is as it is is named as identity denoted by the symbol "id":}
\begin{center}
    \includegraphics[scale = 0.9,width =2cm]{images/identity.png}
    \center\textbf{identity  (id) }
\end{center}

\textbf{The dance move where our identity square rotates by 90 degrees clockwise  shall be called as r1.}
\begin{center}
    \includegraphics[scale = 0.9, width = 2cm]{images/ninty_clockwise.png}
    \center\textbf{rotation of identity by 90 degrees  (r1)}
\end{center}

\textbf{The move in which our identity square rotates by 180 degrees clockwise as r2} 
\begin{center}
    \includegraphics[scale = 0.9, width = 2cm]{images/one_eighty.png}
    \center\textbf{rotation of identity by 180 degrees (r2)}
\end{center}

\textbf{Rotation of identity by 270 degrees in the clockwise direction as r3 }
\begin{center}
    \includegraphics[scale = 0.9, width = 2cm]{images/two_seventy.png}
    \center\textbf{rotation of identity by 270 degrees (r3)}
\end{center}

\pagebreak
\textbf{Reflection of identity on a vertical line ( reflection on y axis) as f \textsubscript{v}}
\begin{center}
    \includegraphics[scale = 0.9,width = 2cm] {images/y_axis.png}
    \center\textbf{Vertical refelction of square (reflection on y axis)  as f\textsubscript{v}}
\end{center}

\textbf{Reflection of identity on a horizontal line  reflection on x axis as f\textsubscript{h}}
\begin{center}
    \includegraphics[scale = 0.9, width = 2cm] {images/x_axis.png}
    \center\textbf{Horizontal reflection of square (reflection on the x axis) as f\textsubscript{h}}
\end{center}
 
\textbf{Rotation of the square along its main diagonal as f\textsubscript{d}}
\begin{center}
    \includegraphics[scale = 0.9, width = 2cm] {images/diagonal.png}
    \center\textbf{Diagonal rotation of square as f\textsubscript{d}}
\end{center}

\textbf{Counter diagonal rotation of the square as f\textsubscript{c}}
\begin{center}
    \includegraphics[scale = 0.9, width = 2cm] {images/counter_diagonal.png}
    \center\textbf{Counter diagonal rotation of square as f\textsubscript{c}}
\end{center}
\pagebreak
\textbf{Now let's see how the given eight dance moves of square help us in verifying at the four axioms groups (closure,associativity,identity,inverse). The real action begins when we start composing dance moves which is the action of performing one dance move after another.On composing two dance moves we find out that we get another dance move
    We learnt that groups preserve these axioms under the application of binary operation (+,-,*,/) on its elements. Here the binary operation we are going to perform is composition which is performing one dance move after another. Let's denote composition by the dot symbol '.'.Our set currently contains all the dance moves a square can perform
    Let's call our dance set as DS. DS contains all the moves of a square \{id,r1,r2,r3,f\textsubscript{v},f\textsubscript{h},f\textsubscript{c},f\textsubscript{d} \}}

\textbf{ One more thing to keep in mind while performing the composition operation on the set is that if we take two elements say r1 and r3 and apply composition to it mathematically it looks like this
    r1 . r3 ("Apply symmetry r1 after performing r3"). we always read composition operation from  FROM RIGHT TO LEFT. That is in this case the square first performs the r3 move ( 270 degrees rotation)  followed by  r2 move (180 degrees rotation.)}


\textbf{We shall now construct a table where we have all the  compositions of the square.   }
\pagebreak
\begin{figure}
    \centering
    \subfloat[\centering label ] {{\includegraphics[width=7cm]{./images/all_composition_notation.png}}}
    \subfloat[\centering label ] {{\includegraphics[width=7cm]{./images/all_composition_actions.png}}}
    \caption{All compositions of symmetries of a square }
\end{figure}
\textbf{Let's see what the two tables mean. First the green coloured  column denotes all the elements of the set DS Similarly the blue coloured row contains all the elements of DS . The small circle on the top represents the composition operator. All the remaining rows and column which are in black shows the result of one composition action followed by another .
    For instance if we look at the third column of the second row we see r1 written. This means that on performing id (id on the blue row) followed by r1 (on the green column ) we get the result r1 . Mathematically that can be written as (r1 . id = r1 ) (remember the rule to read composition from RIGHT TO LEFT we performed identity first followed by r1 )
Taking a random instance say column five of row four we again see r1 which is that result of composing two moves r2 and r3 ( r3 . r2 ). All the remaining elements shows the result of composing one action followed by another. The table on the right just shows how the squares actually move and it is made for the reader to get an intuition and play around with it }

\pagebreak
\section * {FIELDS AND THEIR ADDITIONAL PROPERTIES}
\hrule 
\bigskip
\textbf{We looked at our first algebric structures i.e., Groups and also went through a geometric proof of what the five propositions  were and how they preservered structural integrity on applying a binary operation to elements present inside it
Now we look at Fields which satisfy and preserve all the prepositions(alongwith a few more propositions) which were satisfied by a Group strictly for addition and multiplication i.e, Fields could be written as  (G,"+","*"). There are additional  propositions that fields satisfy alongwith the proposition of groups. Few of the additional propostions that fields satisfy apart from the propositions that group satisfies are discussed ahead.
Before jumping into these additional properties we observe that Fields are an extension of groups. i.e.,Any algebric structure which is a Field inherently preserves the properties of a group as well. So while verifying a field it's always a good practice to see whether the algebraic structure withholds the inherent properties of a group under binary operation. Once that is established further verifications for field can be carried out.We now look at the additional properties that field satisfies
}
\\
1) \textit{Commutative property:\quad}\textbf{ The commutative :wproperty states that given a set of numbers say \{1,2,3,4,5,6\} or any other finite set  if we take two element 'a' and another element 'b' and add it a+b the result should always be equal to the result in the case where i take 'b' first follwed by 'a' i.e, (a+b) should equal (b+a) . Also we know that fields hold the property for both addition and multiplication (a*b) should also equal (b*a).
    Let's take a set {0,1,-1} and apply commutative property for our 'a' and 'b' let's say a=0 and b= 1 a+b = 0+1 = 1 = b+a = 1+0 = 1 (we see commutative property for '+' sign holds) similarly for multiplication a= 0 and b=1 (a*b) = (0*1) =0 and (1* 0) = 0 . We see that both in case of addition as well as in the case of multiplication changing the order of our elements, objects doesn't change the required result 
while performing both addition and multiplication therefore commutative property holds}.
\\
\\
2) \textit{Additive Identity:\quad}\textbf{The additive identity implies presence of 0 in the set  so that on adding any element from the set with 0 we get the same element . For instance in the given set {1,-1,0} since there is a zero the additive identity satisfies if we take any element and add 0 to it (1+0 =1 , 0+0 = 0 and -1+0 = -1) we get back the same.This verifies the presence of  an additive identity in the given set.}
\\
\\
3) \textit{Multiplicative Identity:\quad }\textbf{ The multiplicative identity implies presence of 1 in the set so that on multiplying any element from the set with 1 we get the same element. For instance in the given set \{-1,1,0\} since there is 1, the multiplicative identity satisfies if we take any element and multiply 1 to it \{0*1 = 0 , 1*1 =1 , -1 *1 = -1\} we get back the same element.  }


4)\textit{Additive Inverse:\quad}\textbf{The additive inverse property implies that if ‘a is in the set then , ‘-a’ is in the set such that a+(-a) = 0 In the set \{-1,0,1\} we see that we have additive inverse of -1 which is 1   we have additive inverse of 1 which is -1 and additive inverse of 0 is 0 itself \{1+(-1)=0, 0+(-0) = 0, -1+(1) = 0\}}
\\
\\
5)\textit{Multiplicative Inverse:\quad}\textbf{Multiplicative inverse of any number (except the number 0) is that number which when multiplied with the original number gives us one. For example let a number from our set be 'a' On multiplying 'a' with it's multiplicative inverse the result is always 1 ( a * (multiplicative inverse of a) ) = 1 . The elements of  set\{0,1,-1\} except 0 (since there is no multiplicative inverse of 0 ) satisfies all the condition 
(1 * (1) =1 , -1 * (-1) = 1) . 1 *(multiplicative inverse of 1) = 1
and -1 *(multiplicative inverse of -1) = 1}
\\
\\

6)\textit{Distributive property:\quad}\textbf{The distributive identity states that if operations like a * (b+c) gives the same output as (a*b)+(a*c) . This property is is one property where additon and multiplication are involved together and the output is true in both the equation. Let's take our set \{-1,0,1\} to understand this  
n performing -1*(0+1) we get -1 which is basically performing (-1*0)+(-1*1) = -1 . Distributive property is inherently preserved for a Field under addition and multiplication.}
\\
\\
\pagebreak 
\section * {ARITHMETIC IN TERMS OF MODULO }
\hrule
\bigskip
\textbf{Every reader reading this booklet is very much aware of conventional system of addition, subtraction, multiplication and division of two numbers which is very trivial
\\
\\1+1 = 2, 1-1 = 0 , 1*1 = 1,  1/1 = 1
\\1728+1 =1729, 1000-999= 1, 17*3= 51,10896601/3301 = 3301
}
\\ 
\\
\textbf{In this section however we are going to carry out these basic operations in terms of modulo. This is the correct time for us to be aware about what modulo actually is .
\\In simple terms a modulo can be understood as  the remainder we get while dividing two numbers. 
\\
Just as  addition is denoted  by '+' sign ,Subtraction by '-' sign Multiplication by '*' sign and Division by '/' sign. Modulo is represented by \%\ sign
\\ Let us look at few examples 
\\ 7 \%\ 2 = 1 ( the remainder when 7 is divided by 2 =1 so 7 \%\ 2 could be written as 1)
\\ 8 \%\ 2 = 0 ( the remainder when 8 is divided by 2 = 0 so 8 \%\ 2 could be written as 0)
\\ 131 \%\ 13 = 1 ( the remainder when 131 is divided by 13 = 1 so 131 \%\ 13 = 1)
\\
Modulo can be thought of as a "wraparound" or "clock" math. The answers to questions like "It is currently 3 o'clock what hour will it be 47 hours from now?" can be answered using modulo arithmetic}
\\
\\

\includegraphics[scale = 0.25,width =9cm]{images/clock.png}
\textbf{To answer the question on previous page we use the modulo operator 
currently it is 3 and we are asked time 47 hours from now. For carrying out this we first add add 47 to 3 (47+3) and then take modulo by 12 since any the value of "hour" has to lie between [0,11] strictly \\
the above  answer  is calculated by (3+47)\%\ 12 = 2 . }

\textbf {Any operation between two variables could be re-written in terms of modulo.}
\\
\
\textbf{Given a modulo number "m" ( in our wrapped clock case m was 12) the operations addition, subtraction,multiplication and division could be re-written this way}
\\
\\
\textbf{ a + b  becomes equivalent to (a+b)\%\ m in case of modulo addition}
\\
\\
\textbf { a - b becomes equivalent to (a-b)\%\ m in case of modulo subtraction }
\\
\\
\textbf { a * b becomes equivalent to (a*b)\%\ m in case of modulo multiplication}
\\
\\
\textbf { a / b becomes equivalent to (a/b)\%\ m in case of modulo division }

\pagebreak 
\section * {FINITE FIELD SETS}
\hrule
\bigskip
\textbf{ We learnt about Groups and Sets  in the  first section followed by different axioms and propositions that were inherently satisfed by  set elements under application of operators like '+','-','*' and '/' .Now we look at finite fields which are a type of fields that strictly preserve certain  properties (closed property,additive identity,multiplicative identity,additive inverse,multiplicative inverse,commutative property and distributive property under modulo arithmetic  )}.
\\
\\
\textbf{ In case of a finite field if the size of set is p, all the objects,elements inside the set are strictly from 0 to p-1.\\ \\ Mathematically the finite field set looks like this \\ \\
F(p) = \{0,1,2,3....p-1\} (Read as "a finite field of order  p")
\\ \\
F(p) is called the "Finite field of p" (why finite? because the set has finite elements starting from 0 to p-1 (one less than p)
\\\\ 
F(29) = "Finite field of 29"
F(17) = "Finite field of 17"
the "p" in F(p) is always a prime number 
\\ \\
A finite field of order 11 looks like this \\
F(11) = \{0,1,2,3,4,5,6,7,8,9,10\} \\
A Finite field of order 13 looks like this \\
F(13) = \{0,1,2,3,4,5,6,7,8,9,10,11,12\}
\\
\\
For understanding elliptic curve cryptography which is at the heart of bitcoin transaction we consider and understand finite fields whose order is prime.
}
\pagebreak 
\section * {APPLYING MODULO ARITHMETIC TO FINITE FIELD ELEMENTS}
\hrule
\bigskip 
\textbf{Now we apply the two concepts we read earlier i.e., finite field sets and modulo arithmetic to prove that each element in our finite field preserve the  propositions  of sets we've discussed earlier ( closure property ,additive identity,multiplicative identity,multiplicative inverse,additive inverse,commutative property, distributive property)
}
\\
\textbf{Let us consider finite field of 11 i.e., F(11) = \{0,1,2,3,4,5,6,7,8,9,10\} and apply modulo arithmetic to prove the properties. In the case of group we verified proofs geometrically using a symmetries of square. Here we use raw number and add a pinch of mathematical rigor while proving the proprties.
\\ 
\\
Take any two number from F(11) and add it say take a = 10 and b = 9. We take the modulo with the prime number "p" or the field order 
\\
\\
on addition (10+9) \%\ 11 = (19) \%\ 11 = 9 (which is inside the field)
\\
\\
on subtraction (10-9) \%\ 11 = (1) \%\ 11 = 1  (which also lies inside the field)
\\
\\
on multiplication (10*9) \%\ 11 = (90) \%\ 11 = 2 (which also lies inside the field)}
\\
\\
\textbf{Note:}\textit{We will discuss division on finite fields  while proving the multiplicative inverse property for finite fields} 
\\
\\
\textbf{On taking any two elements from the field and applying first three three binary operators ('+','-','*') with modulo arithmetic for evaluation we see that our closure property satisfies ( which is take any two members from set apply binary operations (we've seen '+','-', and '*') with modulo arithmetic  and the result we get is strictly inside the field.}

\pagebreak 
\section * {APPLYING MODULO ARITHMETIC TO FINITE FIELD ELEMENTS }
\hrule 
\bigskip 
\textbf{The second and third property of  additive identity and multiplicative identity holds by default as we always have 0 and 1 as our elements in our field no matter what set we consider F( of any p) always has 0 and 1 in it  and ‘a’+0 = ‘a’ and ‘a’*1 =’a’ these two cases are always satisfied}
\\
\\
\textbf{Let's check for the commutative property. Consider 'a' as 5 and 'b' as 6 on adding (a+b) \%\ p we get (5+6) \%\ 11 = 0 . Now on exchanging b with a (6+5)\%\ 11 =0. We can try out with each and every number from our finite fields and see that commutative property satisfies.
}
\\
\\
\textbf{Similarly the distribuitive property implicitly gets  preserved on this is one example which shows that (6*(5+9))\%\ 11 =7  which is same as (6*5+9*5)\%\ 11 = 7}
\\
\\
\textbf{ Let's check for the additive inverse property i.e., on choosing any 'a' from the finite field element we get 'a'+(-'a') = 0 
Let's verify it on F(11) = \{0,1,2,3,4,5,6,7,8,9,10\} 
\\
1+(-1) \%\ 11 = 0 \\
2+(-2) \%\ 11 = 0 \\
3+(-3) \%\ 11 = 0 and so on for all members }
\\
\textbf{The final property i.e., multiplicative inverse property is one of the most counter intuitive property we prove. At a first glance it is very non obvious if we go by the definition i.e. we multiply a number from the field with its inverse to get 1 . Let's take 2 as our number now we need to find such a number which exists in the field and on multiplication gives us 1.The only number we could think is (1/2) as 2*(1/2) = 1 . But we see 1/2 is not present in the field itself. So how do we get the intuition? .The answer to the question is given by Fermat's little theorem. Let's look at the theorem.}
\pagebreak 
\section * {FERMAT'S LITTLE THEOREM}
\hrule 
\bigskip
\textbf{Let's us consider  a finite field of any prime order say  F(11) = \{0,1,2,3,4,5,6,7,8,9,10\} \\
Then we  multiply each number in the given set by a number say 'n' (n is strictly greater than  0 and less or equal to p-1). Take  n=1, n=2 and n=3 (multiplication here refers to modulo multiplication (n*2) will be (n*2)\%\ p) 
\\ In the first case when we multiply each element by 1 (we exclude 0 since our n should be greater than 0 and less than or equal to p-1 so our set starts from 1) the resultant set looks something like this \\
\{1 \%\ 11, 2 \%\ 11, 3 \%\ 11, 4\%\ 11, ...... 10 \%\ 11 \}
\\
In the second case when we multiply each element by 2 the resultant set looks something like this \\ 
\{(2*1) \%\ 11 , (2*2) \%\ 11, (2*3) \%\ 11, (2*4) \%\ 11, .... (2*10) \%\ 11 \} 
\\
In the third case when we multiply each element by 3 the resultant set looks something like this \\ 
\{ (3*1) \%\ 11, (3*2) \%\ 11, (3*3) \%\ 11, (3*4) \%\ 11, ....(3*10) \%\ 11 \} 
\\
On evaluating each set the results we get can be seen \\
For n = 1 our final set looks like this \{ 1,2,3,4,5,6,7,8,9,10\} \\ 
For n = 2 our final set looks like this \{
2,4,6,8,10,1,3,5,7,9\} \\
For n = 3 our final set looks like this \{
3,6,9,1,4,7,10,2,5,8\} \\
\\
On carefully examining each set we get we can see that each set just contains all the original elements we started with ( if we write the sets we obtained by multiplying by n=2 and n=3 in ascending order we still get the original set i.e. \{1,2,3,4,5,6,7,8,9,10\} 
\\
\\
From here we can conclude that on modulo multiplication of our field elements by any 'n' which is greater than 0 and less than p-1  we get the original field element. This statement can be mathematically written as \\ \\
\{1 \%\ p, 2 \%\ p, 3 \%\ p, 4\%\,p ......(p-1) \%\ p \} = \{(n*1) \%\ p, (n*2) \%\ p, (n*3) \%\ p ......(n* p-1) \%\ p \} for any n which is greater than 0 
\\
}
\pagebreak
\section * {FERMAT'S LITTLE THEOREM}
\hrule 
\bigskip 
\{1 \%\ p, 2 \%\ p, 3 \%\ p, 4\%\,p ......(p-1) \%\ p \} = \{(n*1) \%\ p, (n*2) \%\ p, (n*3) \%\ p ......(n* p-1) \%\ p \}
\\
\\
\textbf{Since  same numbers are in
both sets. We can then multiply every element in both sets to get this equality
\\
\\
(1*2*3*4*5*6*7...p-1) \%\ p = (n*2n*3n*4n*..(p-1)*n) \%\ p
\\
\\
On the left side the term (1*2*3*4*5....p-1) can be written as factorial of p-1 (factorial of a number = number * (number-1) * (number -2) .... 1 e.g., factorial of 2 = 2*1 factorial of 3 = 3*2*1 Factorial is represented by the symbol "!" ) \\
On the right side  (1*2*3*4*5....p-1) again  can be written as factorial of p-1 and since we have  multiplied the term 'n'  p-1 times on the right side  (n*n*n.. p-1 times) that term becomes n raised to power p-1 \\ \\ 
our equation now becomes:} 
\\ 
\\ 
\textbf{\[ (p-1)! \%\ p = ((p-1)! * n^{p-1}) \%\ p   \] }
\textbf {The (p-1)! terms on left hand side and right hand side gets cancelled out and our final expression becomes as :}
\\
\\
\textbf{\[ 1 \%\ p = (n ^{p-1}) \%\ p \]}
\\
\\
\textbf{Which is the Fermat's little theorem}

\pagebreak
\textbf{We've proved the Fermat's little theorem on the previous page but the entire idea  of doing so is to see how multiplicative inverses are calculated inside the field 
\\
Another important bit of information that we have to keep in mind is that we've not yet proved the closure criteria for division (modulo division) of two finite field elements.Proving closure under division using Fermat's theorem will give us the perfect premise for understanding multiplicative inverse. So let's use the Fermat's theorem to prove closure property of finite fields under division (i.e., take any two 'a' and 'b' from field and perform (a/b) \%\ p the result still lies inside the field)}
\\
\textbf{Let's consider our old friend field F(11) = \{0,1,2,3,4,5,6,7,8,9,10\} and take any two numbers say 'a' = 5 and 'b' = 6 and perform modular division (5/6) \%\ 11. Intuitively we see that there is no way that the result lies on the field itself. One thing to note here is that the operation (a/b) can also be written as a * b inverse (power of b on the denominator is 1 when it comes to numerator the power becomes -1)}
\textbf{\[((a^{1}/b^{1})\%\ p) = (a^{1}* b ^{-1}) \%\ p \]}
\textbf{We know what the value of 'a'   but we don't know what the value of b inverse  is so here we use the Fermat's Theorem to find it. Rewriting the Fermat's theorem  }
\\
\textbf{\[ 1 \%\ p = (n ^{p-1}) \%\ p \]}
\textbf{ In the above term 1 \%\ p can be simply written as 1 }
\textbf{\[ 1  = (n ^{p-1}) \%\ p \]}
\textbf{Also b inverse can be written as b inverse * 1 (any number can be written as number * 1)}
\textbf{\[ (b^{-1}) =  (b^{-1} *1)  \]}
\\
\textbf{we can replace the value of 1 in the above equation by the value of 1 we have obtained in Fermat's little theorem. On doing so  }
\textbf{ \[ b^{-1} =(b^{-1} *  n ^{p-1})  \%\ p \] }
\textbf{Now since we know n is any number greater than 0 and less than p-1. Therefore  n could be b  as well or n could be a or  n could be replaced by any number greater than 0 and less than or equal to p-1.On replacing n by b we get }
\pagebreak
\textbf {\[ (b^{-1}) = (b^{-1} * b^{p-1}) \%\ p \]  }
\textbf{we know when two bases are same power can be added ( a power p * a power p can be written a  power (p+p) similarly  here on the right hand side of the above equation  b power -1 and b power p-1 can be written as b power -1+p-1 so rewriting the equation we have }
\textbf {\[ (b^{-1}) = (b ^{-1 + (p-1) = p-2}) \%\ p \]}
\textbf {Finally the result we get for b inverse is }
\textbf {\[ (b^{-1}) = (b ^{p-2})\%\ p \] }
\textbf {Above we have the value of b inverse that we have obtained using Fermat's little theorem now we come back to our original problem which we started with that is modulo division of finite field. We had taken our a as 5 and b as 6, the field under consideration was F(11) p = 11. (a/b) \%\ p could be written as (a * b inverse) \%\ 11 . Now that we know how to find the value of b inverse let's find it using the above equation}
\textbf {\[ 6 ^{-1}  = 6 ^{11 -2} \%\ p \]}
\textbf{\[6 ^{-1} = 6 ^{9} \%\ p \]}
\textbf {\[6 ^{-1} = 10077696 \%\ 11 \]}
\textbf {\[6 ^{-1} = 2\]} 
\textbf {\[ (5/6) \%\ 11 = (5 * 6 ^{-1} ) \%\ 11 \]}
\textbf {\[ (5/6) \%\ 11 = (5 * 2 ) \%\ 11 \]}
\textbf {\[ (5/6) \%\ 11 = 10 \]}
\textbf { We see that the result we obtained above (10) lies inside the field. This might not be super-intuitive at the first place but on circling back to the derivations and proofs the ideas start to make sense. Also the whole idea of doing this was to find out the multiplicative inverse which magically we've already done i.e., the value of b inverse we've found out in the above equation is the way the multiplicative inverse of any number from the field. We shall now be showing a few examples how we take  a number and multiplying it with the multiplicative inverse gives us 1 ( this was the criteria for multiplicative identity) }

\pagebreak 
\section * {EXAMPLES}
\textbf {Given a finite field F(11) = \{1,2,3,4,5,6,7,8,9,10\} prove multiplicative identity for 4 also show  (7/8) lies inside the field}
\\
\textbf {Let's prove multiplicative identity first}
\textbf {Let b = 4. We find b inverse first We know that :}
\textbf{\[b ^{-1} = (b ^{p-2})\%\ p \]}
\textbf{\[4 ^{-1} = (4 ^{9}) \%\ 11 \]}
\textbf{\[4^{-1} = 262144 \%\ 11 \]}
\textbf{ \[4^{-1} = 3 \]}
\textbf{ Now that we've found out that inverse of 4 is 3 to prove the multiplicative identity we must show that (a * a inverse) \%\ p  should strictly be 1 and 'a' shouldn't be 0 as multiplicative inverse of 0 does not exist}
\textbf{(4 * 3 ) \%\ 11 = (12) \%\ 11 = 1  which proves the multiplicative identity.Similarly we can prove it for 5 and 6 as well and any number in the field}
\textbf{ Now we see that modulo division follows closed property ( i.e., (a/b) \%\ p lies inside the field itself. Let's take the example of (7/8) }

\textbf {\[(7/8) = (7 * 8^{-1})\]}
\textbf { Since we know that }
\textbf {\[b^{-1} = b ^{p-2} \%\ p \]}
\textbf {\[8 ^{-1} = 8 ^{11-2} \%\ 11 \]}
\textbf {\[8 ^{-1} = 134217728 \%\ 11 \]}
\textbf {\[8 ^{-1} = 7 \]}
\\
\\
\textbf {\[(7* 7) \%\ 11 = 5\]}
\textbf { we see that on division we get 5 which strictly lies inside the field you can now take other examples and verify for yourself.Now that we have verified closure( modulo addition, subtraction, multiplication, division of any two finite field elements always lies inside field),additive identity (a + 0 = 0 ), multiplicative identity (a* 1 = a), additive inverse (a +(-a) = 0) and multiplicative inverse ( a* a inverse =1),we are now ready to look forward to a new topic. }

\pagebreak





\end{document}






